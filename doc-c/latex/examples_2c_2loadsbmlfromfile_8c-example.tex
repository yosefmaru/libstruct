\hypertarget{examples_2c_2loadsbmlfromfile_8c-example}{}\section{examples/c/loadsbmlfromfile.\+c}
This is an example of how to load a S\+B\+ML file and print structural analysis test results.


\begin{DoxyCodeInclude}
\textcolor{preprocessor}{#include <stdio.h>}           \textcolor{comment}{// for printf}
\textcolor{preprocessor}{#include <stdlib.h>}          \textcolor{comment}{// for malloc}
\textcolor{preprocessor}{#include <string.h>}          \textcolor{comment}{// for memset}
\textcolor{preprocessor}{#include <\hyperlink{libstructural_8h}{libstructural.h}>}   \textcolor{comment}{// the structural analysis library}


\textcolor{keywordtype}{int} main (\textcolor{keywordtype}{int} argc, \textcolor{keywordtype}{char}** argv)
\{
        \textcolor{keywordtype}{int}             result;
        \textcolor{keywordtype}{char}*   message;
        \textcolor{keywordtype}{int}             length;
        
        \textcolor{keywordflow}{if} (argc < 2)
        \{
                printf(\textcolor{stringliteral}{"please provide a filename as argument"});
                \textcolor{keywordflow}{return} -1;
        \}
        
        \textcolor{comment}{// load the sbml file and check the argument}
        result = \hyperlink{libstructural_8cpp_a8fe9803fda8adddfd10c7d48f6fdbfe1}{LibStructural\_loadSBMLFromFile}(argv[1], &message, &length);
        \textcolor{keywordflow}{if} (result != 0)
        \{
                printf(\textcolor{stringliteral}{"the SBML file %s could not be loaded."}, argv[1]);
                \textcolor{keywordflow}{return} -1;
        \}
        
        \textcolor{comment}{// print model overview}
        printf(\textcolor{stringliteral}{"%s"}, message);
        
        \textcolor{comment}{// free the memory used by the message}
        \hyperlink{libstructural_8cpp_a43ec36f000082b338e1aad9cff3b0b23}{LibStructural\_freeVector}(message);
        
        \textcolor{comment}{// obtain and print the test results}
        \hyperlink{libstructural_8cpp_a76fa7cdfb5333db9e78bf36ac6bc292e}{LibStructural\_getTestDetails}( &message, &length );
        printf(\textcolor{stringliteral}{"%s"}, message);
        
        \textcolor{comment}{// finally free the memory used by the message}
        \hyperlink{libstructural_8cpp_a43ec36f000082b338e1aad9cff3b0b23}{LibStructural\_freeVector}(message);
        
        
        \textcolor{keywordflow}{return} 0;
\}

\textcolor{comment}{// Output for model BorisEJB.xml(available with SBW distribution) passed in }
\textcolor{comment}{//}
\textcolor{comment}{//-----------------------------------------------------------------------------}
\textcolor{comment}{//-----------------------------------------------------------------------------}
\textcolor{comment}{//STRUCTURAL ANALYSIS MODULE : Results }
\textcolor{comment}{//-----------------------------------------------------------------------------}
\textcolor{comment}{//-----------------------------------------------------------------------------}
\textcolor{comment}{//Size of Stochiometric Matrix: 8 x 10 (Rank is  5)}
\textcolor{comment}{//Nonzero entries in Stochiometric Matrix: 20  (25% full)}
\textcolor{comment}{//}
\textcolor{comment}{//Independent Species (5) :}
\textcolor{comment}{//MKK\_P, MAPK\_P, MKKK, MKK, MAPK}
\textcolor{comment}{//}
\textcolor{comment}{//Dependent Species (3) :}
\textcolor{comment}{//MKK\_PP, MKKK\_P, MAPK\_PP}
\textcolor{comment}{//}
\textcolor{comment}{//L0 : There are 3 dependencies. L0 is a 3x5 matrix.}
\textcolor{comment}{//}
\textcolor{comment}{//Conserved Entities}
\textcolor{comment}{//1:  + MKK\_P + MKK + MKK\_PP}
\textcolor{comment}{//2:  + MKKK + MKKK\_P}
\textcolor{comment}{//3:  + MAPK\_P + MAPK + MAPK\_PP}
\textcolor{comment}{//-----------------------------------------------------------------------------}
\textcolor{comment}{//-----------------------------------------------------------------------------}
\textcolor{comment}{//Developed by the Computational Systems Biology Group at Keck Graduate Institute }
\textcolor{comment}{//and the Saurolab at the Bioengineering Departmant at  University of Washington.}
\textcolor{comment}{//Contact : Frank T. Bergmann (fbergman@u.washington.edu) or Herbert M. Sauro.   }
\textcolor{comment}{//}
\textcolor{comment}{//(previous authors) Ravishankar Rao Vallabhajosyula                   }
\textcolor{comment}{//-----------------------------------------------------------------------------}
\textcolor{comment}{//-----------------------------------------------------------------------------}
\textcolor{comment}{//}
\textcolor{comment}{//Testing Validity of Conservation Laws.}
\textcolor{comment}{//}
\textcolor{comment}{//Passed Test 1 : Gamma*N = 0 (Zero matrix)}
\textcolor{comment}{//Passed Test 2 : Rank(N) using SVD (5) is same as m0 (5)}
\textcolor{comment}{//Passed Test 3 : Rank(NR) using SVD (5) is same as m0 (5)}
\textcolor{comment}{//Passed Test 4 : Rank(NR) using QR (5) is same as m0 (5)}
\textcolor{comment}{//Passed Test 5 : L0 obtained with QR matches Q21*inv(Q11)}
\textcolor{comment}{//Passed Test 6 : N*K = 0 (Zero matrix)}
\end{DoxyCodeInclude}
 